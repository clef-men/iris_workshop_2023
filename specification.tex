\begin{frame}{Specification --- \texttt{chaselev\_make}}
\small
\centering
\[
	\triplePretty{
		\iTrue
	}{
		\texttt{chaselev\_make}\ \exprUnit
	}{
		\lambda\ t \cdot
		\tikz[baseline, remember picture]{\node[anchor=base] (inv_1) {${
			\textcolor{blue}{\mathrm{chaselev \mathhyphen inv}\ t\ \iota}
		}$}} \iSep
		\tikz[baseline, remember picture]{\node[anchor=base] (model_1) {${
			\textcolor{red}{\mathrm{chaselev \mathhyphen model}\ t\ []}
		}$}} \iSep
		\tikz[baseline, remember picture]{\node[anchor=base] (owner_1) {${
			\textcolor{teal}{\mathrm{chaselev \mathhyphen owner}\ t}
		}$}}
	}
\]
\vfill
\only<2>{
	\tikz[remember picture] \node[align=left] (inv_2) {
		enforces a protocol (using an Iris invariant)
	} ;
	\begin{tikzpicture}[remember picture, overlay]
		\path[draw=blue, thick, ->] (inv_2.north) to [out=100, in=-100] (inv_1.south) ;
	\end{tikzpicture}
}
\only<3>{
	\tikz[remember picture] \node[align=left] (model_2) {
		asserts the list of values that the deque (logically) contains
	} ;
	\begin{tikzpicture}[remember picture, overlay]
		\path[draw=red, thick, ->] (model_2.north) to [out=100, in=-100] (model_1.south) ;
	\end{tikzpicture}
}
\only<4>{
	\tikz[remember picture] \node[align=left] (owner_2) {
		gives the owner exclusive access to his end of the deque
	} ;
	\begin{tikzpicture}[remember picture, overlay]
		\path[draw=teal, thick, ->] (owner_2.north) to [out=100, in=-100] (owner_1.south) ;
	\end{tikzpicture}
}
\end{frame}

% ---------------------------------------------------------

\begin{frame}{Specification --- \texttt{chaselev\_push}}
\centering
\[
	\atripleExtPretty{
		\tikz[baseline, remember picture]{\node[anchor=base] (inv_1) {${
			\textcolor{blue}{\mathrm{chaselev \mathhyphen inv}\ t\ \iota}
		}$}} \iSep
		\tikz[baseline, remember picture]{\node[anchor=base] (owner_1) {${
			\textcolor{red}{\mathrm{chaselev \mathhyphen owner}\ t}
		}$}}
	}{
		vs
	}{
		\tikz[baseline, remember picture]{\node[anchor=base] (model_1) {${
			\textcolor{teal}{\mathrm{chaselev \mathhyphen model}\ t\ vs}
		}$}}
	}{
		\texttt{chaselev\_push}\ t\ v
	}{
		\uparrow\iota
	}{
	}{
		\tikz[baseline, remember picture]{\node[anchor=base] (model_2) {${
			\textcolor{teal}{\mathrm{chaselev \mathhyphen model}\ t\ (vs \mdoubleplus [v])}
		}$}}
	}{
		\exprUnit
	}{
		\tikz[baseline, remember picture]{\node[anchor=base] (owner_2) {${
			\textcolor{red}{\mathrm{chaselev \mathhyphen owner}\ t}
		}$}}
	}
\]
\begin{overbox}<2>
	\small
	\centering
	Specification of a concurrent operation ($\simeq$ transaction):
	
	standard triple + logically atomic triple
	
	\[
		\atripleExtPretty{
			\textcolor{red}{P}
		}{
			\textcolor{purple}{\overline{x}}
		}{
			\textcolor{teal}{P_\mathrm{lin}}
		}{
			e
		}{
			\mathcal{E}
		}{
			\textcolor{purple}{\overline{y}}
		}{
			\textcolor{teal}{Q_\mathrm{lin}}
		}{
			res
		}{
			\textcolor{red}{Q}
		}
	\]
	\begin{align*}
			\textcolor{red}{P}
			&:
			\text{private precondition}
		\\
			\textcolor{red}{Q}
			&:
			\text{private postcondition}
		\\
			\textcolor{teal}{P_\mathrm{lin}}
			&:
			\text{public precondition}
		\\
			\textcolor{teal}{Q_\mathrm{lin}}
			&:
			\text{public postcondition}
	\end{align*}
\end{overbox}
\begin{overbox}<3>
	For a concurrent data structure:
	
	\[
		\atripleExtPretty{
			\textcolor{blue}{\mathrm{??? \mathhyphen inv} \cdots} \iSep
			\textcolor{red}{P}
		}{
			\textcolor{purple}{\overline{x}}
		}{
			\textcolor{teal}{\mathrm{??? \mathhyphen model} \cdots}
		}{
			e
		}{
			\mathcal{E}
		}{
			\textcolor{purple}{\overline{y}}
		}{
			\textcolor{teal}{\mathrm{??? \mathhyphen model} \cdots}
		}{
			res
		}{
			\textcolor{red}{Q}
		}
	\]
\end{overbox}
\vfill
\only<4>{
	\tikz[remember picture] \node[align=left] (owner_3) {
		owner's operation
	} ;
	\begin{tikzpicture}[remember picture, overlay]
		\path[draw=red, thick, ->] (owner_3.north) to [out=100, in=-100] (owner_1.south) ;
		\path[draw=red, thick, ->] (owner_3.north) to [out=100, in=-100] (owner_2.south) ;
	\end{tikzpicture}
}
\only<5>{
	\tikz[remember picture] \node[align=left] (model_3) {
		$v$ is atomically pushed at the owner's end
	} ;
	\begin{tikzpicture}[remember picture, overlay]
		\path[draw=teal, thick, ->] (model_3.north) to [out=100, in=-100] (model_1.south) ;
		\path[draw=teal, thick, ->] (model_3.north) to [out=100, in=-100] ([xshift=1cm]model_2.south) ;
	\end{tikzpicture}
}
\end{frame}

% ---------------------------------------------------------

\begin{frame}{Specification --- \texttt{chaselev\_pop}}
\footnotesize
\centering
\[
	\atripleExtPretty{
		\tikz[baseline, remember picture]{\node[anchor=base] (inv_1) {${
			\textcolor{blue}{\mathrm{chaselev \mathhyphen inv}\ t\ \iota}
		}$}} \iSep
		\tikz[baseline, remember picture]{\node[anchor=base] (owner_1) {${
			\textcolor{red}{\mathrm{chaselev \mathhyphen owner}\ t}
		}$}}
	}{
		vs
	}{
		\tikz[baseline, remember picture]{\node[anchor=base] (model_1) {${
			\textcolor{teal}{\mathrm{chaselev \mathhyphen model}\ t\ vs}
		}$}}
	}{
		\texttt{chaselev\_pop}\ t
	}{
		\uparrow\iota
	}{
		o
	}{
		\tikz[baseline, remember picture]{\node[anchor=base] (model_2) {${
			\textcolor{teal}{
				\bigvee\left[\begin{array}{l}
						vs = [] \iSep
						o = \exprNone \iSep
						\textcolor{teal}{\mathrm{chaselev \mathhyphen model}\ t\ []}
					\\
						\exists v, vs' \cdot
						vs = vs' \mdoubleplus [v] \iSep
						o = \exprSome{v} \iSep
						\textcolor{teal}{\mathrm{chaselev \mathhyphen model}\ t\ vs'}
				\end{array}\right]
			}
		}$}}
	}{
		o
	}{
		\tikz[baseline, remember picture]{\node[anchor=base] (owner_2) {${
			\textcolor{red}{\mathrm{chaselev \mathhyphen owner}\ t}
		}$}}
	}
\]
\vfill
\only<2>{
	\tikz[remember picture] \node[align=left] (owner_3) {
		owner's operation
	} ;
	\begin{tikzpicture}[remember picture, overlay]
		\path[draw=red, thick, ->] (owner_3.north) to [out=100, in=-100] (owner_1.south) ;
		\path[draw=red, thick, ->] (owner_3.north) to [out=100, in=-100] (owner_2.south) ;
	\end{tikzpicture}
}
\only<3>{
	\tikz[remember picture] \node[align=left] (model_3) {
		either 1) the deque is seen empty \\
		or 2) some value $v$ is atomically popped at the owner's end
	} ;
	\begin{tikzpicture}[remember picture, overlay]
		\path[draw=teal, thick, ->] (model_3.north) to [out=100, in=-100] (model_1.south) ;
		\path[draw=teal, thick, ->] (model_3.north) to [out=100, in=-100] ([xshift=1cm]model_2.south) ;
	\end{tikzpicture}
}
\end{frame}

% ---------------------------------------------------------

\begin{frame}{Specification --- \texttt{chaselev\_steal}}
\footnotesize
\centering
\[
	\atripleExtPretty{
		\tikz[baseline, remember picture]{\node[anchor=base] (inv_1) {${
			\textcolor{blue}{\mathrm{chaselev \mathhyphen inv}\ t\ \iota}
		}$}}
	}{
		vs
	}{
		\tikz[baseline, remember picture]{\node[anchor=base] (model_1) {${
			\textcolor{teal}{\mathrm{chaselev \mathhyphen model}\ t\ vs}
		}$}}
	}{
		\texttt{chaselev\_steal}\ t
	}{
		\uparrow\iota
	}{
		o
	}{
		\tikz[baseline, remember picture]{\node[anchor=base] (model_2) {${
			\textcolor{teal}{
				\bigvee\left[\begin{array}{l}
						vs = [] \iSep
						o = \exprNone \iSep
						\mathrm{chaselev \mathhyphen model}\ t\ []
					\\
						\exists v, vs' \cdot
						vs = v :: vs' \iSep
						o = \exprSome{v} \iSep
						\mathrm{chaselev \mathhyphen model}\ t\ vs'
				\end{array}\right]
			}
		}$}}
	}{
		o
	}{
		\iTrue
	}
\]
\vfill
\only<2>{
	\tikz[remember picture] \node[align=left] (model_3) {
		either 1) the deque is seen empty \\
		or 2) some value $v$ is atomically popped at the thieves' end
	} ;
	\begin{tikzpicture}[remember picture, overlay]
		\path[draw=teal, thick, ->] (model_3.north) to [out=100, in=-100] (model_1.south) ;
		\path[draw=teal, thick, ->] (model_3.north) to [out=100, in=-100] ([xshift=1cm]model_2.south) ;
	\end{tikzpicture}
}
\end{frame}